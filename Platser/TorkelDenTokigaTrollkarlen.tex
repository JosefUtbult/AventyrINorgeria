\chapter*{Magiker med relationsproblem}
%
\section*{Torkel den Tokiga Trollkarlen}
Torkel den Tokiga Trollkarlen är en gammal man av ursprungligen alv-ras. Han bor i en liten stuga norr om Håggesund.
%
\subsection*{Torkels hus}
Torkels hus ligger två dagar norr om Håggesund. Hans hus är ett ganska stort trähus, men ganska slitet. Där inne är det fullt med olika saker och målningar på väggarna på olika alkemiska cirklar.
I mitten av rummet står ett fat med vatten och en lapp där det står något. \\ \\
%
\textit{``Är jag inte här så är det nog något som gått fel. Kan du vara så snäll att doppa en tå i fatet? Fördelaktigen en Alv-tå. Tack på förhand''} \\ \\
%
Om någon doppar en tå materialiseras en trollkarl i ett rökmoln som säger poff. Trollkarlen blir av samma ras som tån som doppats, och blir ganska lik individen i fråga, men av manligt kön och gammal. Man blir uppskattad av Torkel om han blivit alv, men helt ok med att blivit människa. Han blir dock ganska grinig om han blivit gnom, halvman eller dvärg. Är han grinig är han inte jättetaggad på att prata, men han kan bli övertalad om man är lite karesmatisk eller bjuder på mat eller vin.
%
\section*{Torkels uppdrag}
Torkel har ett uppdrag till de resande. Hans ärkerival \textit{Haggan Agda, Hardals Häxa} har stulit hans slott, \textit{Hardals Fästning}, och tvingat honom bo i hans gamla håla, som han tidigare använt som förråd. Nu sitter han här och är grinig mestadels av tiden. Det äventyrarna kan göra för Torkel är att ta med en liten lila behållare med någon tjock vätska i till Hardals fästning och lura i Agda denna vätska. Den måste konsumeras och Agda får inte komma till onödig skada. Som belöning kan Torkel betala äventyrarna med 40 guldmynt, som han har i slottet. Frågar de något om Agda verkar han lite sur på henne men påpekar att hon är välldigt gästvänlig och gärna bjuder in resande på en måltid och plats över natten.
%
\section*{Hardals Fäsning}
Hardals fästning ligger fem kilometer norr om Torkels hem, i dalen Hardel. Fästningen vaktas av goblins som verkar vara anställda av Agda. De resande kan försöka smita sig in i borgen, vilket gör Agda misstänksam om de hittas. De kan trasig in med våld och de kan knacka på och be om att få komma in. Agda släpper gärna in dem om de inte är allt för sketchy. Om de ber om det bjuder hon på mat i matsalen och sovplats över natten i gästrummen.

Slottet har en liten innergård, två våningar och ett torn. I tornet är Agdas arbetsrum. På första våningen finns en hall, ett kök med mathiss, två gästrum och arbetarnas levnadsutrymme (hängmattor och lite bord). På andra våningen finns Agdas sovrum och en kombinerad salong samt matsal. 
%
\section*{Haggan Agda, Hardals Häxa}
Agda är en kort, gammal människokvinna klädd i en rob som släpar efter henne. Hon är också en magiker som spenderar mestadels av sin tid i sitt arbetsrum och sin salong. Hon talar gärna med främlingar. Om någon frågar om hennes relation med Torkel nämner hon att den inte är så bra för tillfället. Hon säger att de föll isär för några år sedan vilket ledde till deras skilsmässa. Hon vill inte höra av Torkel mer efter det.

Om Äventyrarna får i Agda vätskan ser ni hur hon börjar darra. Stora bölder växer ut ur hela hennes kropp som växer till lämmar och ett huvud. Det slutar med att Agda förvandlats till Torkel, som snabbt trollar fram lite kläder till sig själv. Han tackar äventyrarna och hämtar deras guld åt dem. Goblinserna är något fientliga mot Torkel då Agda salt åt dem att hålla honom borta, men han påpekar att det är han som nu har allt guld och kan betala dem vilket mildrar dem. Vill de sova över natten så går det bra för Torkel då de hjälpt honom, men sedan vill han att de ska ut. 

\subsection*{Hallen}
Hallen är ganska stor och öppen med en trappa längst med motsatt vägg från ingången. Den har två dörrar på vänster sida till vardera gästrum och en på högersidan till arbetarnas levnadsutrymme. I ett hörn står en byrå och i ett annat står en uppstoppad björn. 

\subsection*{Gästrummen}
Gästrummen innehåller en stor säng, en eldstad och en byrå vardera. I ett av rummen finns en tavla med en gammal alv-kvinna på. 

\subsection*{Arbetarnas levnadsutrymme}
Arbetarnas levnadsutrymme är goblinarnas rum. Det sitter fyra goblins runt bordet och två ligger i varsin hängmatta. Det finns tio hängmattor hängda längst väggar och upphissade i taket och ett runt bord med stolar. Det finns även en lista med höftskynke i och något kopparmynt. Runt väggarna står två spjut, två svärd och tre armborst med varsit hölster.

\subsection*{Salongen}
I salongen finns ett stort bord med stolar och en soffgrupp runt en eldstad. Bokhyllor står längst med alla väggar som är fyllda med diverse böcker. Över eldstaden hänger ett magiskt svärd. 

\subsection*{Arbetsrummet}
% TODO: Skriv om arbetsrummet
%
\section*{Agdas uppdrag}
Hela uppdraget kan göras igen, men nu med Agda som uppdragsgivare. Hon bor i Torkels stuga nu och ger dem en påse med torkade örter och en glasflaska med olja i. Örterna ska konsumeras av Torkel och efter det ska han smörjas i oljan. Vid konsumtion blir Torkel en groda, och när grodan smörja blir denna Agda (med kläder). Torkel är dock inte villig att släppa in äventyrarna självmant då han ser igenom deras trick.
