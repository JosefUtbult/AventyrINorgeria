\chapter*{Tramsø - Den vita staden}
%
Tramsø är en av de två, och den enda allmänt kända alvstaden i Norgeria. Tramsø bebos näst intill endast av överalver, med ytterst få lärda samt köpmän av andra raser. Tramsø består av en 200 meter hög mur, vit som snö, med själva staden på toppen. Exakt vad för teknik och magi som användes är för länge sedan bortglömt, men man vet att det för hundratusentals år sedan byggdes en mur runt ett berg, berget plattades till och muren fylldes upp för att skapa en cylindrisk platå på vilken staden byggdes. Staden är nästintill ointagbar då de enda vägarna upp är smala branta gångar längst med muren som är tungt försvarade.

Invånarna i Tramsø menar att staden styrs som en teknokrati, vilket det en gång varit, men efter hundratals år har de statliga organen stagnerat till att styras av en grupp ledarmöten som sitter på sin post livet ut: \textit{Rådet}. Dessa väljs in av resterande medlemmar och brukar allt som oftast vara arvtagare av de före detta ledamöterna. Rådet styr från \textit{Universitetet}, som ligger som centrum för staden.
