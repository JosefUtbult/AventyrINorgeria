\subsubsection{Alver}
Alver är sällsynta i Norgeria, men det finns fästen för dem där de lever i exkluderande samhällen. De har ofta svårt att erkänna det men de ser på andra raser med avsmak. Det finns två större fästen, Tramsø, den vita staden och Jord I Rana, underjordens hemlighet. 
%
\paragraph{Alvernas skriftspråk}
De moderna alverna har två skriftspråk; Alvtext och Uråldrig Alv-skrift. Alvtext är det som används i modern tid och detta kan alla alver, samt vissa lärda, läsa och skriva. Uråldrig Alv-skrift är däremot ett utdött språk. Detta kan finnas i ruiner och på fornlämningar. Det är ett utdött språk, men det har återupptäcktes och översatts av högt uppsatt lärda alver. Det kan tydas till viss del, men det är oerhört få alver som är lärda inom detta.
%
\paragraph{De fördömdas sigil}
Alver har under tusentals år haft en egen vållad förbannelse över dem som ärvs vidare i generationer. Denne kallas De fördömdas sigill, som är en magisk formel ditsatt av uråldriga förfäder av de moderna alverna. De fördömdas sigil beskrivs enligt legenderna som följande.
%
\begin{displayquote}
Den alv som mördar en annan alv, eller direkt bidrar till hans död, blir för evigt bränd med de fördömdas sigil.  
\end{displayquote}
%
De märke som uppstår är en tatuering av ett runt sigill. Detta sigill innefattar skript som beskriver omständigheter runt alvens mord på i alvernas uråldriga skriftspråk.
%
Att ha de fördömdas sigil anses oerhört tabu, då mördandet av en alv ses som det värsta brottet som kan genomföras enligt alver. Alver med ett sigill blir disowned av alla andra alver och utstötta ur alvsammhället. 
%
Ett undantag till detta är alver som fått ett sigill i tjänst av sitt alvrike, då överhuvudet av riket tatuerat in sitt märke i form av sitt förlåtande runt sigillet för att deras synd, enligt dem, ska nollställas. Dessa blir inte disowned, men många alver ser fortfarande ner på alver med denna markering. 

% TODO: Skriv med om alvernas fall